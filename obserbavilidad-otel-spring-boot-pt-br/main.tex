% !TeX encoding = UTF-8
\documentclass[aspectratio=169]{beamer}
\useoutertheme[progressbar=frametitle]{metropolis}
\useinnertheme{metropolis}
\definecolor{nabgray}{rgb}{0.6,0.59,0.61}
\usecolortheme[named=nabgray]{structure}
\usepackage{fontawesome5}
\usepackage{tikz}
\usepackage[utf8]{inputenc}
\usepackage[brazilian]{babel}
\usepackage{fontspec}
\setmonofont{JetBrains Mono}
\setmainfont{Helvetica Neue}
\setsansfont{Helvetica Neue}

\usepackage{smartdiagram}
\usepackage{qtree}
\usepackage{verbatim}
\usepackage{svg}
\usepackage{graphicx}
\usepackage{color}

\definecolor{lightgray}{rgb}{0.95, 0.95, 0.95}
\definecolor{darkgray}{rgb}{0.4, 0.4, 0.4}
%\definecolor{purple}{rgb}{0.65, 0.12, 0.82}
\definecolor{editorGray}{rgb}{0.95, 0.95, 0.95}
\definecolor{editorOcher}{rgb}{1, 0.5, 0} % #FF7F00 -> rgb(239, 169, 0)
\definecolor{editorGreen}{rgb}{0, 0.5, 0} % #007C00 -> rgb(0, 124, 0)
\definecolor{orange}{rgb}{1,0.45,0.13}
\definecolor{olive}{rgb}{0.17,0.59,0.20}
\definecolor{brown}{rgb}{0.69,0.31,0.31}
\definecolor{purple}{rgb}{0.38,0.18,0.81}
\definecolor{lightblue}{rgb}{0.1,0.57,0.7}
\definecolor{lightred}{rgb}{1,0.4,0.5}
\usepackage{upquote}
\usepackage{listings}
\lstset{language=java,
	basicstyle=\footnotesize\ttfamily,
	keywordstyle=\footnotesize\color{blue}\ttfamily,
	escapeinside={<@}{@>}
}

\usebackgroundtemplate%
{%
	\includegraphics[width=\paperwidth]{Images/Contenido}%
}

\title{OpenTelemetry: A língua franca da observabilidade}
\subtitle{Você não pode melhorar o que não pode medir}
\author{Nabenik}
\date{\today}

\begin{document}
	
	{
		\usebackgroundtemplate{\includegraphics[width=\paperwidth]{Images/portada}}
		\begin{frame}
			\maketitle
		\end{frame}
	}
	
	\begin{frame}{Observabilidade e monitoramento}
		
		\begin{columns}
			
			\begin{column}{0.5\textwidth}
				Sistema de TI {\Huge \faServer}
				\begin{itemize}
					\item Aplicações
					\item Serviços de terceiros
					\item Processos intermédios
					\item Hosts
				\end{itemize}
			\end{column}
			
			\begin{column}{0.5\textwidth}
				Fontes de informação {\Huge	 \faBinoculars}
				\begin{itemize}
					\item Eventos
					\item Logs
					\item Métricas
					\item Traces
				\end{itemize}
			\end{column}
		\end{columns}
		Teoria: Coletamos dados, processamos para identificar KPIs (golden signals) e criamos alarmes 
	\end{frame}


\begin{frame}{Observabilidade e monitoramento}
	
			Sistema de TI {\Huge \faServer}
			\begin{itemize}
				\item Aplicações: Binario (Go, GraalVM Native), minified (JS), jar, war (Java), K8s (containers), FaaS
				\item Serviços de terceiros (PaaS especializados)
				\item Processos intermédios (Banco de dados, message queue)
				\item Hosts (LSB, Alpine, openrc, Windows, etc.)
				\item ElasticSearch, Grafana, Datadog, New Relic, Cloudwatch, etc.
			\end{itemize}
	
\end{frame}

\begin{frame}{Observabilidade e monitoramento}
	
	Fontes de informação {\Huge	 \faBinoculars}
	\begin{itemize}
		\item Logs: Arquivos (/var/log), Systemd, FluentD
		\item Metrics: Hosts, JMX, Prometheus endpoints (Spring Actuator, Microprofile Metrics) 
		\item Traces: Jaeger, Zipkin
	\end{itemize}
	
\end{frame}

\begin{frame}{Observabilidade e monitoramento}
		Precisamos dados de \textbf{fontes e formatos diversos}, os quais podem ser \textbf{enviados ou coletados} para ser analisados em \textbf{plataformas de observabilidade} com diversas capacidades e riscos -e.g. Vendor lock in, tecnologias muito especificas, incompatibilidade entre versões- 
	
	\begin{alertblock}{Precisamos ...}
	data pipelines de observabilidade baseados numa \textbf{especificação} padrão
	\end{alertblock}
	
\end{frame}
	
%	{
%		\usebackgroundtemplate{\includegraphics[width=\paperwidth]{Images/separador}}
%		\setbeamercolor{normal text}{fg=white}
%		\setbeamercolor{frametitle}{fg=red}
%		\usebeamercolor[fg]{normal text}
%		\section{Fontes de dados}
%	}
%	
%	\begin{frame}{Logs}
%		
%		\begin{block}{Definição}
%			Registro histórico de processos, eventos e mensagens com dados adicionais como timestamps e contexto
%		\end{block}	
%		Modo clássico
%		\begin{itemize}
%			\item Arquivos no sistema
%			\item Systemd
%			\item FluentD
%		\end{itemize}	
%	\end{frame}
%	
%	\begin{frame}
%		\begin{figure}
%			\centering
%			\includegraphics[width=0.8\linewidth]{Images/logs-sample}
%			\label{fig:logs-sample}
%		\end{figure}
%	\end{frame}
%	
%	\begin{frame}{Métricas}
%		
%		\begin{block}{Definição}
%			Valores numéricos sobre o desempenho das aplicações -e.g. Desempenho da rede, tamanho do heap-
%		\end{block}	
%		Formatos
%		\begin{itemize}
%			\item Gauge (instantâneos)
%			\item Delta (diferença entre dois pontos)
%			\item Acumulativas (mudanças ao longo do tempo)
%		\end{itemize}
%		
%		Modo clássico
%		\begin{itemize}
%			\item Sistema operacional
%			\item JMX
%			\item Endpoint HTTP (/actuator/metrics) - ServiceMonitor
%		\end{itemize}	
%	\end{frame}
%	
%	\begin{frame}
%		\begin{figure}
%			\centering
%			\includegraphics[width=0.7\linewidth]{Images/metrics-sample}
%			\label{fig:metrics-sample}
%		\end{figure}
%	\end{frame}
%	
%	\begin{frame}{Trace}
%		
%		\begin{block}{Definição}
%			Caminho completo de uma operação dentro de uma série de sistemas inter-relacionados. Cada participante cria spans que juntos formam os traces
%		\end{block}	
%		Modo clássico
%		\begin{itemize}
%			\item Protocolos push proprietários -e.g. Jaeger, Zipkin-
%		\end{itemize}	
%	\end{frame}
%	
%	\begin{frame}
%		\begin{figure}
%			\centering
%			\includegraphics[width=0.7\linewidth]{Images/trace-sample}
%			\label{fig:trace-sample}
%		\end{figure}
%	\end{frame}
%	
	{
		\usebackgroundtemplate{\includegraphics[width=\paperwidth]{Images/separador}}
		\setbeamercolor{normal text}{fg=white}
		\setbeamercolor{frametitle}{fg=red}
		\usebeamercolor[fg]{normal text}
		\section{OpenTelemetry}
	}
	
	
	\begin{frame}
		\begin{figure}
			\centering
			\includegraphics[width=0.8\linewidth]{Images/datalake}
			\label{fig:datalake}
			\caption{Data pipeline}
		\end{figure}
	\end{frame}
	
	
	\begin{frame}
		
		\begin{columns}
			\begin{column}{0.75\textwidth}
				\begin{figure}
					\centering
					\includegraphics[width=\linewidth]{Images/otelcollector}
					\label{fig:otelcollector}
				\end{figure}
			\end{column}
			\begin{column}{0.25\textwidth}  %%<--- here
				\begin{enumerate}
					\item Instrumentação
					\item Coleta
					\item Envio
				\end{enumerate}
			\end{column}
		\end{columns}
		
		
	\end{frame}
	
	\begin{frame}
		\begin{figure}
			\centering
			\includegraphics[width=\linewidth]{Images/landscape}
			\label{fig:cncfprojects}
		\end{figure}
	\end{frame}
		
	\begin{frame}{Instrumentação}
		
		\begin{itemize}
			\item {\faCode} \textbf{Biblioteca (Library)}  
			\begin{itemize}
				\item Instrumentação manual usando APIs do OpenTelemetry diretamente no código
			\end{itemize}
			
			\item {\faCogs} \textbf{Framework}  
			\begin{itemize}
				\item Integração com frameworks que já oferecem suporte oficial
				\item Ex: Spring Boot, Quarkus, ASP.NET Core
			\end{itemize}
			
			\item {\faMagic} \textbf{Manipulação de artefato}  
			\begin{itemize}
				\item Instrumentação automática por meio de interceptores, agentes ou manipulação de bytecode
				\item Ex: Java Agent (javaagent), AWS Lambda layer, JavaScript Zero code instrumentation
			\end{itemize}
		\end{itemize}
		
	\end{frame}
	
	
	\begin{frame}{Otel Collector - Upstream}
		
		\begin{figure}
			\centering
			\includegraphics[width=0.7\linewidth]{Images/otelupstream}
			\label{fig:otelupstream}
		\end{figure}
		
		
	\end{frame}
	
	\begin{frame}
		
		\begin{figure}
			\centering
			\includegraphics[width=\linewidth]{Images/oteldistros}
			\label{fig:oteldistros}
		\end{figure}
		
		
	\end{frame}
	
	
	
	\begin{frame}
		\begin{figure}
			\centering
			\includegraphics[width=0.8\linewidth]{Images/elasticotel}
			\label{fig:elasticotel}
		\end{figure}
	\end{frame}
	
	\begin{frame}
\begin{figure}
	\centering
	\label{fig:adotcollector}
	\includegraphics[width=0.8\linewidth]{Images/adotcollector}
\end{figure}
	\end{frame}
	
	
	\begin{frame}{Fatos aleatórios que eu aprendi nesta jornada}
		
		
		\begin{enumerate}
			\item A instrumentação via \textit{agent} é particularmente boa no Java
			\item A instrumentação de framework tem uma limitante, quando você pega bibliotecas não -opinionated- ... ela não funciona
			\item Para alguns frameworks -e.g. Spring Boot- existe a instrumentação oficial do OpenTelemetry e a instrumentação oficial do projeto
			\item A instrumentação via \textit{agent} precisa fine-tunning em FaaS,  -e.g. No AWS lambda o boot fica (ainda mais) lento sem tunning-
			\item As \textit{distribuições} do collector facilitam o envio de dados
			\item O barco da \textit{padronização} já esta em alto mar, ele chama-se OpenTelemetry
			\end{enumerate}
	\end{frame}
	
	\begin{frame}
		\begin{figure}
			\centering
			\includegraphics[width=\linewidth]{Images/aiops}
			\label{fig:aiops}
		\end{figure}
	\end{frame}
	
	{
		\usebackgroundtemplate{\includegraphics[width=\paperwidth]{Images/separador}}
		\setbeamercolor{normal text}{fg=white}
		\setbeamercolor{frametitle}{fg=red}
		\usebeamercolor[fg]{normal text}
		\section{Exemplo}
	}
	
	\begin{frame}
		
		
		\begin{columns}
			
			\begin{column}{0.5\textwidth}
		\begin{figure}
			\centering
			\includegraphics[width=\linewidth]{Images/otel}
			\label{fig:otel}
		\end{figure}		
			\end{column}
			
			\begin{column}{0.5\textwidth}
				\begin{figure}
					\centering
					\includegraphics[width=0.7\linewidth]{Images/dzoneotel}
					\label{fig:dzoneotel}
				\end{figure}
\begin{figure}
	\centering
	\includegraphics[width=0.4\linewidth]{Images/QR.png}
	\label{fig:qrdzone}
\end{figure}
			\end{column}
		\end{columns}
		
		
		
	\end{frame}
	\begin{frame}{Víctor Orozco}
		\begin{columns}[T]
			
			\begin{column}[T]{4cm}
				\begin{figure}
					\centering
					\includegraphics[width=0.8\linewidth]{Images/logos}
				\end{figure}
			\end{column}
			\begin{column}[T]{6cm}
				\begin{itemize}
					\item vorozco@nabenik.com
					\item \href{https://twitter.com/tuxtor}{@tuxtor}
					\item \href{https://vorozco.com}{https://vorozco.com}
					\item \href{https://tuxtor.shekalug.org}{https://tuxtor.shekalug.org}
				\end{itemize}
				\begin{center}
					\includegraphics[width=0.1\linewidth]{Images/cclogo}
					\\
					Este trabalho está licenciado sob a licença Creative Commons Atribuição-NãoComercial-CompartilhaIgual 3.0 Guatemala (CC BY-NC-SA 3.0 GT).
				\end{center}
			\end{column}
		\end{columns}
	\end{frame}

	
\end{document}
