% !TeX encoding = UTF-8
\documentclass{beamer}
\usetheme{metropolis}           % Use metropolis theme
\usepackage{tikz}
\usepackage[utf8]{inputenc}
\usepackage[spanish]{babel}

\usepackage{smartdiagram}
\usepackage{qtree}
\usepackage{listings}
\lstset{language=Java,
                basicstyle=\footnotesize\ttfamily,
                keywordstyle=\footnotesize\color{blue}\ttfamily,
}
%\usetheme{Hannover}
%\usecolortheme{crane}
\usepackage{graphicx}
\usepackage{tikz}
\usepackage{hyperref}
\usetikzlibrary{mindmap,backgrounds}

\definecolor{lightgray}{rgb}{0.95, 0.95, 0.95}
\definecolor{darkgray}{rgb}{0.4, 0.4, 0.4}
%\definecolor{purple}{rgb}{0.65, 0.12, 0.82}
\definecolor{editorGray}{rgb}{0.95, 0.95, 0.95}
\definecolor{editorOcher}{rgb}{1, 0.5, 0} % #FF7F00 -> rgb(239, 169, 0)
\definecolor{editorGreen}{rgb}{0, 0.5, 0} % #007C00 -> rgb(0, 124, 0)
\definecolor{orange}{rgb}{1,0.45,0.13}		
\definecolor{olive}{rgb}{0.17,0.59,0.20}
\definecolor{brown}{rgb}{0.69,0.31,0.31}
\definecolor{purple}{rgb}{0.38,0.18,0.81}
\definecolor{lightblue}{rgb}{0.1,0.57,0.7}
\definecolor{lightred}{rgb}{1,0.4,0.5}

%\usebackgroundtemplate%
%{%
%	\includegraphics[width=\paperwidth]{Images/Contenido}%
%}
\title{Desmitificando la inteligencia artificial}
\author{Víctor Orozco}
\institute{Nabenik}
\date{\today}

\begin{document}

\frame{\titlepage}

\section{Introducción}


\begin{frame}{Víctor Orozco}
    \begin{columns}[T] % contents are top vertically aligned
        \begin{column}[T]{5cm} % each column can also be its own environment
            \begin{itemize}
                \item Java Champion
                \item Ex becario OAS-GCUB
                \item \href{https://www.oracle.com/javaone/dukes-choice-award.html}{Dukes Choice Award 2016 -GuateJUG-}
                \item \href{http://www.nabenik.com}{CTO/Founder -Nabenik-}
                \item \href{https://twitter.com/tuxtor}{@tuxtor}
                \item \href{http://vorozco.com}{The J*} 
            \end{itemize}
        \end{column}
        \begin{column}[T]{3cm} % alternative top-align that's better for graphics
            \begin{figure}
                \centering
                \includegraphics[width=\linewidth]{Images/logos}
            \end{figure}
            
        \end{column}
    \end{columns}
\end{frame}

\subsection{Inteligencia Artificial}

\begin{frame}{Inteligencia Artificial}
    \begin{itemize}
        \item \textbf{Entender y construir} entidades 
        inteligentes.
        \item Primeros pasos en robótica
        \item Programas que puedan/sepan reaccionar ante incertezas (CS)
    \end{itemize}
\end{frame}

\begin{frame}{Ramas clásicas}
    \begin{center}
        \begin{tikzpicture}[scale=0.6, transform shape]
        \path[mindmap,concept color=blue,text=white,
        level 1 concept/.append style=
        {every child/.style={concept color=blue!70},sibling angle=-30}]
        node[concept] {IA}
        [clockwise from=0]
        child[concept color=red] { node[concept] {Procesamiento de lenguajes} }
        child[concept color=blue] { node[concept] {Aprendizaje de maquina y mineria de datos}}
        child[concept color=orange] { node[concept] {Visión por computador}}
        child[concept color=red] { node[concept] {Planeamiento} }
        child[concept color=blue] { node[concept] {Representación del conocimiento} }
        child[concept color=orange] { node[concept] {Razonamiento y toma de decisiones} }
        child[concept color=red] { node[concept] {Strong AI} }
        child[concept color=blue] { node[concept] {Robótica} };
        \end{tikzpicture}
    \end{center}
\end{frame}

\begin{frame}{Ramas clásicas}
    \begin{center}
        \begin{tikzpicture}[scale=0.6, transform shape]
        \path[mindmap,concept color=blue,text=white,
        level 1 concept/.append style=
        {every child/.style={concept color=blue!70},sibling angle=-30}]
        node[concept] {IA}
        [clockwise from=0]
        child[concept color=red] { node[concept] {Procesamiento de lenguajes} }
        child[concept color=orange] { node[concept] {Aprendizaje de maquina y mineria de datos}}
        child[concept color=red] { node[concept] {Visión por computador}}
        child[concept color=red] { node[concept] {Planeamiento} }
        child[concept color=red] { node[concept] {Representación del conocimiento} }
        child[concept color=red] { node[concept] {Razonamiento y toma de decisiones} }
        child[concept color=red] { node[concept] {Strong AI} }
        child[concept color=red] { node[concept] {Robótica} };
        \end{tikzpicture}
    \end{center}
\end{frame}


\begin{frame}{Ramas clásicas}

\begin{center}
	\includegraphics[width=0.7\linewidth]{Images/data}
\end{center}

\end{frame}


\subsection{Motivación}
\begin{frame}
    \begin{center}
        \large ¿Por qué?
    \end{center}
\end{frame}
\begin{frame}{Motivación}
    \begin{center}
        \includegraphics[width=\linewidth]{Images/ml1}
    \end{center}
\end{frame}

\begin{frame}{Motivación}
    \begin{center}
        \includegraphics[width=\linewidth]{Images/ml2}
    \end{center}
\end{frame}

\begin{frame}{Motivación}
    \begin{center}
        \includegraphics[width=\linewidth]{Images/ml4}
    \end{center}
\end{frame}

\begin{frame}{Motivación}
    \begin{center}
        \includegraphics[width=\linewidth]{Images/ml5}
    \end{center}
\end{frame}

\section{Aprendizaje}
\begin{frame}
    \begin{center}
        \large Mejores predicciones
    \end{center}
\end{frame}

\begin{frame}{Inferencia}
    \begin{center}
        \includegraphics[width=0.9\linewidth]{Images/inferencia}
    \end{center}
\end{frame}

\begin{frame}{Inferencia}
\begin{itemize}
    \item Estadística inferencial (Excel, BI)
    \begin{itemize}
        \item Regresión de datos
        \item Redes bayesianas
    \end{itemize}
    \item \textbf{Aprendizaje de maquina} (Sistemas de recomendación, chatbots)
    \begin{itemize}
        \item Perceptrones
        \item Redes neurales
        \item Clustering
        \item KNN
        \item SNA
    \end{itemize}
\end{itemize}
\end{frame}

\begin{frame}
    \begin{center}
        \large Mejores predicciones
    \end{center}
    \begin{itemize}
        \item Venta (Chatbots, sistemas de recomendación)
        \item Fidelización (Software consciente de contexto, análisis de redes sociales)
        \item Producción (Redes neurales, redes bayesianas)
        \item Análisis (Map-Reduce (aka Big Data))
    \end{itemize}
\end{frame}

\begin{frame}{Netflix}
    \begin{center}
        \includegraphics[width=0.9\linewidth]{Images/netflix}
    \end{center}
\end{frame}

\begin{frame}{Inferencia (retos)}
    \begin{itemize}
        \item Problema
        \item Modelo
        \item Implementación
    \end{itemize}
\end{frame}

\subsection{Modelo}
\begin{frame}{1-2-3 Machine Learning}
    \begin{enumerate}
        \item Normalizar los datos
        \item Crear el modelo
        \item Entrenar el modelo
        \item Comprobar su funcionamiento
    \end{enumerate}
\end{frame}

\begin{frame}{Que}
    \begin{itemize}
        \item Probabilidad
        \item Estructura
        \item Conceptos ocultos (Hidden concepts)
    \end{itemize}
\end{frame}

\begin{frame}{Donde}
    \begin{itemize}
        \item Supervised learning (objetivo)
        \item Unsupervised learning (conceptos ocultos)
        \item Reinforcement learning (feedback)
    \end{itemize}
\end{frame}

\begin{frame}{Para que}
    \begin{itemize}
        \item Predicciones
        \item Diagnostico
        \item Sumarizaciones
    \end{itemize}
\end{frame}

\begin{frame}{Como}
    \begin{itemize}
        \item Pasivo (Observador)
        \item Activo
        \item Offline
        \item Online
    \end{itemize}
\end{frame}

\begin{frame}{Salida}
    \begin{itemize}
        \item Clasificación (Binario)
        \item Regresión (Continuo)
    \end{itemize}
\end{frame}

\begin{frame}{Detalles}
    \begin{itemize}
        \item Generativo (Generalizaciones)
        \item Discriminativo (Distinguir)
    \end{itemize}
\end{frame}

\begin{frame}{Navaja de Occam}
    \begin{center}
        \includegraphics[width=0.7\linewidth]{Images/occam}
    \end{center}
\end{frame}

\begin{frame}{Navaja de Occam}
    "Pluralitas non est ponenda sine necessitate"
    
    "Plurality is not to be posited without necessity"
\end{frame}

\begin{frame}{Navaja de Occam (Español)}
    Cuando se tienen dos teorias que obtenen las mismas predicciones, generalmente la más simple es la mejor
\end{frame}



\subsection{Implementación}
\begin{frame}{Bibliotecas}
    Principales
    \begin{itemize}
        \item DeepLearning4J \url{https://deeplearning4j.org/}
        \item BID Data Project \url{http://bid2.berkeley.edu/bid-data-project/}
        \item Neuroph \url{http://neuroph.sourceforge.net/index.html}
                \item Smile \url{http://haifengl.github.io/smile/}
    \end{itemize}

    Complementarias
    \begin{itemize}
        \item Commons Math \url{http://commons.apache.org/proper/commons-math/}
        \item Eclipse Collections \url{https://www.eclipse.org/collections/}
    \end{itemize}
\end{frame}

\begin{frame}{Paas}
    \begin{itemize}
        \item AmazonML \url{https://aws.amazon.com/machine-learning/}
        \item Bluemix - Watson \url{https://www.ibm.com/cloud-computing/bluemix/watson}
        \item Oracle Advanced Analytics \url{https://www.oracle.com/database/advanced-analytics/index.html}
    \end{itemize}
\end{frame}



\section{Experiencias previas}

\begin{frame}{JRiskSimulator}
    \begin{itemize}
        \item \textbf{Problema:} Mejorar las recomendaciones en ISO 27001
        \item \textbf{Modelo: } Clasificación inmediata mediante análisis de redes sociales
        \item \textbf{Implementación: }  JGraph + JUNG + Commons Math + Java FX
    \end{itemize}
\end{frame}

\begin{frame}{JRiskSimulator}
    \begin{center}
        \includegraphics[width=0.9\linewidth]{Images/jrisk}
    \end{center}
\end{frame}

\begin{frame}{Medmigo}
    \begin{itemize}
        \item \textbf{Problema:} Adaptar la recomendación de un profesional de acuerdo a las recomendaciones de mis amigos
        \item \textbf{Modelo: } Clasificación inmediata mediante perceptrones + Análisis de redes sociales
        \item \textbf{Implementación: } Neuroph + Commons Math + Lucene Search + Java EE 
    \end{itemize}
\end{frame}

\begin{frame}{Medmigo}
    \begin{center}
        \includegraphics[width=0.9\linewidth]{Images/Medmigo}
    \end{center}
\end{frame}

\begin{frame}{SGB - Bible Generation}
    \begin{itemize}
        \item \textbf{Problema:} Indexar n cantidad de biblias en un metabuscador que soporte "palabras parecidas"
        \item \textbf{Modelo: } Binary tree + Tokenization + Levenshtein distance + Lazy data fetch
         \item \textbf{Implementación: } NLP + Lucene Search + Java EE 
    \end{itemize}
\end{frame}

\begin{frame}{SGB - Bible Generation}
    \begin{center}
        \includegraphics[width=0.9\linewidth]{Images/bible}
    \end{center}
\end{frame}


\section{Fin}

\begin{frame}{Gracias}
\begin{itemize}
\item me@vorozco.com
\item http://vorozco.com
\item http://github.com/tuxtor/slides
\end{itemize}
\begin{center}
\includegraphics[width=0.1\linewidth]{Images/cclogo}
\\
This work is licensed under a Creative Commons Attribution-ShareAlike 3.0 Guatemala License.
\end{center}
\end{frame}
\end{document}

