\documentclass{beamer}
\useoutertheme[progressbar=frametitle]{metropolis}
\useinnertheme{metropolis}
\definecolor{nabgray}{rgb}{0.6,0.59,0.61}
\usecolortheme[named=nabgray]{structure}

\usepackage{tikz}
\usepackage[utf8]{inputenc}
\usepackage[spanish]{babel}

\usepackage{smartdiagram}
\usepackage{qtree}
\usepackage{verbatim}
\usepackage{svg}
\usepackage{graphicx}
\usepackage{color}

\definecolor{lightgray}{rgb}{0.95, 0.95, 0.95}
\definecolor{darkgray}{rgb}{0.4, 0.4, 0.4}
%\definecolor{purple}{rgb}{0.65, 0.12, 0.82}
\definecolor{editorGray}{rgb}{0.95, 0.95, 0.95}
\definecolor{editorOcher}{rgb}{1, 0.5, 0} % #FF7F00 -> rgb(239, 169, 0)
\definecolor{editorGreen}{rgb}{0, 0.5, 0} % #007C00 -> rgb(0, 124, 0)
\definecolor{orange}{rgb}{1,0.45,0.13}		
\definecolor{olive}{rgb}{0.17,0.59,0.20}
\definecolor{brown}{rgb}{0.69,0.31,0.31}
\definecolor{purple}{rgb}{0.38,0.18,0.81}
\definecolor{lightblue}{rgb}{0.1,0.57,0.7}
\definecolor{lightred}{rgb}{1,0.4,0.5}
\usepackage{upquote}
\usepackage{listings}
\lstset{language=java,
	basicstyle=\footnotesize\ttfamily,
	keywordstyle=\footnotesize\color{blue}\ttfamily,
	escapeinside={<@}{@>}
}
\lstdefinelanguage{Kotlin}{
	comment=[l]{//},
	commentstyle={\color{gray}\ttfamily},
	emph={delegate, filter, first, firstOrNull, forEach, lazy, map, mapNotNull, println, return@},
	emphstyle={\color{purple}},
	identifierstyle=\color{black},
	keywords={abstract, actual, as, as?, break, by, class, companion, continue, data, do, dynamic, else, enum, expect, false, final, for, fun, get, if, import, in, interface, internal, is, null, object, override, package, private, public, return, set, super, suspend, this, throw, true, try, typealias, val, var, vararg, when, where, while},
	keywordstyle={\color{lightblue}\bfseries},
	morecomment=[s]{/*}{*/},
	morestring=[b]",
	morestring=[s]{"""*}{*"""},
	ndkeywords={@Inject, @Deprecated, @JvmField, @JvmName, @JvmOverloads, @JvmStatic, @JvmSynthetic, Array, Byte, Double, Float, Int, Integer, Iterable, Long, Runnable, Short, String},
	ndkeywordstyle={\color{orange}\bfseries},
	sensitive=true,
	stringstyle={\color{olive}\ttfamily},
}


\usebackgroundtemplate%
{%
	\includegraphics[width=\paperwidth]{Images/Contenido}%
}


\title{Enseñando trucos de 20 años a Kotlin}
\author{Víctor Orozco}
\institute{@tuxtor}
\date{\today}

\begin{document}

\frame{\titlepage}

\section{Kotlin}

\begin{frame}{Kotlin}
\begin{columns}
	\begin{column}{0.5\textwidth}
		\begin{itemize}
			\item Tipado estático y compilado
			\item Java inter-op
			\item OO + FP
			\item Null safety
			\item Extensions
			\item Operator overloading
			\item Data classes
			\item One line methods
		\end{itemize}
	\end{column}
	\begin{column}{0.5\textwidth}  %%<--- here
		\begin{figure}
			\centering
			\includegraphics[width=0.7\linewidth]{Images/kotlin}
		\end{figure}
	\end{column}
\end{columns}
\end{frame}

\begin{frame}{Kotlin - Datos interesantes}
\begin{columns}
	\begin{column}{0.5\textwidth}
		\begin{itemize}
			\item Effective Java - Inmutabilidad, builder, singleton, override, final by default, variance by generics
			\item Elvis - Groovy 
			\item Inferencia de tipos - Scala
			\item Inmutabilidad - Scala
			\item Declaración de variables - Scala
			\item Manejo de Null - Groovy
			\item Closures y funciones - Groovy
			\item Google
		\end{itemize}
	\end{column}
	\begin{column}{0.5\textwidth}  %%<--- here
		\begin{figure}
			\centering
			\includegraphics[width=0.7\linewidth]{Images/kotlin}
		\end{figure}
	\end{column}
\end{columns}
\end{frame}

\begin{frame}{Java - Muriendo desde 1995}
\begin{columns}
	\begin{column}{0.5\textwidth}
		\begin{itemize}
			\item Sistemas legados (IBM)
			\item Retrocompatibilidad
			\item Release cadence (6 meses)
			\item Innovación constante en el ecosistema (Spring Boot, Micronaut, MicroProfile, GraalVM)
			\item Raw performance (Beam, Spark, Hadoop)
			\item Tooling - IDE, Maven, Drivers RDBMS
			\item JVM - (Twitter, Alibaba)
			\item OpenJDK
		\end{itemize}
	\end{column}
	\begin{column}{0.5\textwidth}  %%<--- here
		\begin{figure}
			\centering
			\includegraphics[width=0.7\linewidth]{Images/java}
		\end{figure}
	\end{column}
\end{columns}
\end{frame}

\begin{frame}{Java - Muriendo desde 1995}

\begin{figure}
	\centering
	\includegraphics[width=\linewidth]{Images/borges}
\end{figure}
\end{frame}


\begin{frame}{Java - Muriendo desde 1995}

		\begin{figure}
			\centering
			\includegraphics[width=0.7\linewidth]{Images/gossling}
		\end{figure}
\end{frame}


\section{¿Microservicios?}

\begin{frame}{Reactive applications}
Aplicaciones reactivas
\begin{figure}
	\centering
	\includegraphics[width=\linewidth]{Images/reactive-traits}
\end{figure}
Microservicios son una (de muchas) herramienta para creación de aplicaciones reactivas
\end{frame}


\begin{frame}{Microservicios}
\begin{figure}
\centering
\includegraphics[width=\linewidth]{Images/microservicios}
\caption{Microservicios}
\end{figure}
\end{frame}

\begin{frame}{Microservicios - Java}
\begin{itemize}
	\item DIY - Jooby, Javalin, Micronaut, Spark, Vert.x, Helidon SE
	\item Enterprise - Spring Boot, Microprofile (implementaciones)
\end{itemize}
\end{frame}

\begin{frame}{Microservicios - Kotlin}
\begin{itemize}
	\item DIY - Jooby, Javalin, Micronaut, Spark, Vert.x, Helidon SE, \textbf{Ktor}
	\item Enterprise - Spring Boot, Microprofile (implementaciones)
\end{itemize}
\end{frame}


\begin{frame}{}
\begin{figure}
	\centering
	\includegraphics[width=\linewidth]{Images/people}
\end{figure}
\end{frame}

\section{Jakarta EE 8}
\begin{frame}{Jakarta EE 8}
\begin{figure}
	\centering
	\includegraphics[width=\linewidth]{Images/javaee8}
\end{figure}
\end{frame}

\begin{frame}{Jakarta EE 8 - Comunidad Java EE}
\begin{columns}
\begin{column}{0.5\textwidth}
	\begin{figure}
		\centering
		\includegraphics[width=0.7\linewidth]{Images/jakartaee}
	\end{figure}
	\end{column}
	\begin{column}{0.5\textwidth}  %%<--- here
		\begin{figure}
			\centering
			\includegraphics[width=\linewidth]{Images/guardians}
		\end{figure}
	\end{column}
\end{columns}

\begin{figure}
	\centering
	\includegraphics[width=0.7\linewidth]{Images/microprofile-logo}
\end{figure}
\end{frame}

\section{Eclipse MicroProfile}

\begin{frame}{Eclipse MicroProfile}
\begin{figure}
	\centering
	\includegraphics[width=\linewidth]{Images/javaeemicropancake}
	\caption{Credito: Reza Rahman}
\end{figure}
\end{frame}

\begin{frame}{Eclipse MicroProfile}
	\begin{figure}
		\centering
		\includegraphics[width=0.5\linewidth]{Images/oldsetup}
	\end{figure}
\end{frame}

\begin{frame}{Eclipse MicroProfile}
\begin{figure}
	\centering
	\includegraphics[width=\linewidth]{Images/mp5}
\end{figure}
\end{frame}

%TODO 12 fatores

\begin{frame}{Eclipse MicroProfile - Implementaciones}

Bibliotecas
\begin{itemize}
	\item SmallRye (Red Hat)
	\item Hammock
	\item Apache Geronimo
	\item Fujitsu Launcher
\end{itemize}
	
JEAS - Fat Jar, Uber Jar
\begin{itemize}
	\item Dropwizard
	\item KumuluzEE
	\item Helidon (Oracle)
	\item Open Liberty (IBM)
	\item Apache Meecrowave
	\item Thorntail (Red Hat)
	\item Quarkus (Red Hat)
	\item Payara Micro
\end{itemize}
\end{frame}
\begin{frame}{Eclipse MicroProfile - Implementaciones}

Micro server - Thin War
\begin{itemize}
	\item Payara Micro
	\item TomEE JAX-RS
\end{itemize}

Full server
\begin{itemize}
	\item Payara Application Server
	\item JBoss Application Server / Wildfly Application Server
	\item WebSphere Liberty (IBM)
\end{itemize}

https://wiki.eclipse.org/MicroProfile/Implementation
\end{frame}

\section{Eclipse MicroProfile + Kotlin + Maven}

\begin{frame}[fragile]{Eclipse MicroProfile en Payara 5}
\begin{lstlisting}
<dependency>
	<groupId>org.eclipse.microprofile</groupId>
	<artifactId>microprofile</artifactId>
	<type>pom</type>
	<version>2.0.1</version>
	<scope>provided</scope>
</dependency>
\end{lstlisting}
\end{frame}


\begin{frame}[fragile]{Kotlin en Maven - Dependencias}
\begin{lstlisting}[language=XML]
<dependency>
	<groupId>org.jetbrains.kotlin</groupId>
	<artifactId>kotlin-stdlib-jdk8</artifactId>
	<version>${kotlin.version}</version>
</dependency>
\end{lstlisting}
\end{frame}

\begin{frame}[fragile]{Kotlin en Maven - maven-compiler-plugin}
\begin{lstlisting}
<execution>
	<id>default-compile</id>
	<phase>none</phase>
</execution>
<execution>
	<id>default-testCompile</id>
	<phase>none</phase>
</execution>
<execution>
	<id>java-compile</id>
	<phase>compile</phase>
	<goals> <goal>compile</goal> </goals>
</execution>
<execution>
	<id>java-test-compile</id>
	<phase>test-compile</phase>
	<goals> <goal>testCompile</goal> </goals>
</execution>
\end{lstlisting}
\end{frame}


\begin{frame}[fragile]{Kotlin en Maven - kotlin-maven-plugin}
\begin{lstlisting}[
basicstyle=\tiny, %or \small or \footnotesize etc.
]
<compilerPlugins>
<plugin>all-open</plugin>
</compilerPlugins>
...
<option>all-open:annotation=javax.ws.rs.Path</option>
<option>all-open:annotation=javax.enterprise.context.RequestScoped</option>
<option>all-open:annotation=javax.enterprise.context.SessionScoped</option>
<option>all-open:annotation=javax.enterprise.context.ApplicationScoped</option>
<option>all-open:annotation=javax.enterprise.context.Dependent</option>
<option>all-open:annotation=javax.ejb.Singleton</option>
<option>all-open:annotation=javax.ejb.Stateful</option>
<option>all-open:annotation=javax.ejb.Stateless</option>
\end{lstlisting}
\end{frame}



\section{Demo}
\begin{frame}{Kotlin + Jakarta EE + MicroProfile  - Demo}

\begin{itemize}
	\item Kotlin 1.3
	\item Bibliotecas externas - SL4J, Flyway, PostgreSQL
	\item Jakarta EE 8 - EJB, JPA
	\item MicroProfile - CDI, JAX-RS, MicroProfile config
	\item Testing - Arquillian, JUnit, Payara Embedded
\end{itemize}


\normalsize  \url{https://dzone.com/articles/the-state-of-kotlin-for-jakarta-eemicroprofile-tra}\\
\normalsize  \url{https://github.com/tuxtor/integrum-ee}
\end{frame}

\begin{frame}{Kotlin + Jakarta EE + MicroProfile  - Demo}
\begin{figure}
	\centering
	\includegraphics[width=\linewidth]{Images/integrum-ee}
\end{figure}
\end{frame}

\begin{frame}{Kotlin + Jakarta EE + MicroProfile  - Demo}
\begin{figure}
	\centering
	\includegraphics[width=\linewidth]{Images/integrum-deployment}
\end{figure}
\end{frame}


\begin{frame}[fragile]{Kotlin - Entidad JPA}
\begin{lstlisting}[language=Kotlin]
@Entity
@Table(name = "adm_phrase")
@TableGenerator(...)
data class AdmPhrase(
	@Id
	@GeneratedValue(strategy = GenerationType.TABLE,
		generator = "admPhraseIdGenerator")
	@Column(name = "phrase_id")
	var phraseId:Long? = null,
	var author:String = "",
	var phrase:String = ""
)
\end{lstlisting}
Data Clases, Nullable Types
\end{frame}

\begin{frame}[fragile]{Kotlin - Repositorio CDI}
\begin{lstlisting}[language=Kotlin]
@RequestScoped
class AdmPhraseRepository {

	@Inject
	private lateinit var em:EntityManager
	
	...

}
\end{lstlisting}
Lateinit (nullable type)
\end{frame}

\begin{frame}[fragile]{Kotlin - Repositorio CDI}
\begin{lstlisting}[language=Kotlin]
fun create(admPhrase:AdmPhrase) = em.persist(admPhrase)

fun update(admPhrase:AdmPhrase) = em.merge(admPhrase)

fun findById(phraseId: Long) =
em.find(AdmPhrase::class.java, phraseId)

fun delete(admPhrase: AdmPhrase) = em.remove(admPhrase)
. . .
\end{lstlisting}
Single expression functions (One line methods)
\end{frame}

\begin{frame}[fragile]{Kotlin - Repositorio CDI}
\begin{lstlisting}[language=Kotlin]
fun listAll(author: String, phrase: String):
	List<AdmPhrase> {
	
	val query = """SELECT p FROM AdmPhrase p
	where p.author LIKE :author
	and p.phrase LIKE :phrase
	"""
	
	return em.createQuery(query, AdmPhrase::class.java)
		.setParameter("author", "%$author%")
		.setParameter("phrase", "%$phrase%")
		.resultList
}
\end{lstlisting}
Multiline String, mutable declaration
\end{frame}

\begin{frame}[fragile]{Kotlin - Controlador JAX-RS}
\begin{lstlisting}[language=Kotlin, basicstyle=\scriptsize]
@Path("/phrases")
@Produces(MediaType.APPLICATION_JSON)
@Consumes(MediaType.APPLICATION_JSON)
class AdmPhraseController{

	@Inject
	private lateinit var admPhraseRepository: AdmPhraseRepository
	
	@Inject
	private lateinit var logger: Logger
	...

}
\end{lstlisting}
\end{frame}

\begin{frame}[fragile]{Kotlin - Controlador JAX-RS}
\begin{lstlisting}[language=Kotlin, basicstyle=\scriptsize]

@GET
fun findAll(
@QueryParam("author") @DefaultValue("%") author: String,
@QueryParam("phrase") @DefaultValue("%") phrase: String) =
	admPhraseRepository.listAll(author, phrase)

@GET
@Path("/{id:[0-9][0-9]*}")
fun findById(@PathParam("id") id:Long) =
	admPhraseRepository.findById(id)

@PUT
fun create(phrase: AdmPhrase): Response {
	admPhraseRepository.create(phrase)
	return Response.ok().build()
}
\end{lstlisting}
\end{frame}


\begin{frame}[fragile]{Kotlin - Controlador JAX-RS}
\begin{lstlisting}[language=Kotlin, basicstyle=\scriptsize]
@POST
@Path("/{id:[0-9][0-9]*}")
fun update(@PathParam("id") id: Long?, phrase: AdmPhrase)
	:Response {
	if(id != phrase.phraseId) 
		return Response.status(Response.Status.NOT_FOUND).build()
	
	val updatedEntity = admPhraseRepository.update(phrase)
	return Response.ok(updatedEntity).build()
}

@DELETE
@Path("/{id:[0-9][0-9]*}")
fun delete(@PathParam("id") id: Long): Response {
	val updatedEntity = admPhraseRepository.findById(id) ?:
	return Response.status(Response.Status.NOT_FOUND).build()
	admPhraseRepository.delete(updatedEntity)
	return Response.ok().build()
}
\end{lstlisting}
Elvis operator as expression
\end{frame}

\begin{frame}{12 factores cloud native (Heroku)}

\begin{columns}[T] % contents are top vertically aligned
	
	\begin{column}[T]{4cm} % alternative top-align that's better for graphics
		\begin{alertblock}{Microprofile}
	\begin{itemize}
		\item Config
		\item Backing service
		\item Disposability
	\end{itemize}
\end{alertblock}
	\end{column}
	\begin{column}[T]{6cm} % each column can also be its own environment
		\begin{block}{Cloud}
	\begin{itemize}
	\item Codebase (Git-Flow)
	\item Dependencies (Maven)
	\item Build, Release, Run
	\item Processes (Pipelines)
	\item Port binding
	\item Concurrency (Docker - k8s)
	\item Dev / Prod parity
	\item Logs
	\item Admin process
\end{itemize}
\end{block}
	\end{column}
\end{columns}

\end{frame}

\begin{frame}[fragile]{Oracle Cloud}
\begin{lstlisting}[language=XML, basicstyle=\scriptsize]
<groupId>io.fabric8</groupId>
<artifactId>docker-maven-plugin</artifactId>
<version>0.30.0</version>
...
<image>
	<name>iad.ocir.io/tuxtor/microprofile/integrum-ee</name>
	<build>
		<dockerFile>${project.basedir}/Dockerfile</dockerFile >
	</build>
</image>
\end{lstlisting}
\end{frame}

\begin{frame}{Oracle Cloud}
\begin{figure}
	\centering
	\includegraphics[width=0.95\linewidth]{Images/oc1}
\end{figure}
\end{frame}

\begin{frame}{Oracle Cloud}
\begin{figure}
	\centering
	\includegraphics[width=0.95\linewidth]{Images/oc2}
\end{figure}
\end{frame}

\begin{frame}{Oracle Cloud}
\begin{figure}
	\centering
	\includegraphics[width=\linewidth]{Images/oc3}
\end{figure}
\end{frame}

\begin{frame}{Oracle Cloud}

\begin{figure}
	\centering
	\includegraphics[width=\linewidth]{Images/oc4}
\end{figure}
\end{frame}




\begin{frame}{Víctor Orozco}
\begin{columns}[T] % contents are top vertically aligned
	
	\begin{column}[T]{4cm} % alternative top-align that's better for graphics
		\begin{figure}
			\centering
			\includegraphics[width=\linewidth]{Images/logos}
		\end{figure}
	\end{column}
	\begin{column}[T]{6cm} % each column can also be its own environment
		\begin{itemize}
			\item vorozco@nabenik.com
			\item \href{https://twitter.com/tuxtor}{@tuxtor}
			\item \href{http://www.nabenik.com}{http://www.nabenik.com}
		\end{itemize}
	\begin{center}
		\includegraphics[width=0.1\linewidth]{Images/cclogo}
		\\
		This work is licensed under a Creative Commons Attribution-ShareAlike 3.0.
	\end{center}
	\end{column}
\end{columns}
\end{frame}


\end{document}

s